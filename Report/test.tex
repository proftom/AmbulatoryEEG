\documentclass[]{article}

%packages
\usepackage[english]{babel}
\usepackage[nottoc,notlof,notlot]{tocbibind} % Put the bibliography in the ToC
\usepackage{hyperref} % use hyperlinked ToC
\usepackage{parskip}
\usepackage{booktabs} % for much better looking tables
\usepackage{array} % for better arrays (eg matrices) in maths
\usepackage{paralist} % very flexible & customisable lists (eg. enumerate/itemize, etc.)
\usepackage{verbatim} % adds environment for commenting out blocks of text & for better verbatim
\usepackage{subfig} % make it possible to include more than one captioned figure/table in a single float
\usepackage{listings} %for code listings
\usepackage{color} %for colored syntax highligting
\usepackage{rotating}
\usepackage{pdflscape}
\usepackage[]{algorithm2e}
\usepackage{multirow}
\usepackage{float}
\usepackage{mathtools}
\usepackage{amssymb}
\usepackage{geometry} % to change the page dimensions

%page geometry
\geometry{a4paper} % or letterpaper (US) or a5paper or....
\geometry{margin=2.5cm}

%paragraphs and text 
\setlength{\parindent}{4em}
\setlength{\parskip}{4em} %this doesn't seem to be bloody working! Hack it with "\\\\" 
\renewcommand{\baselinestretch}{1.2}
\setlength{\parskip}{10pt plus 1pt minus 1pt}





\title{A Wireless  Low Energy Ambulatory Electroencephalogram}
\author{Thomas Alexander Morrison (tm1810)}
\begin{document}

\maketitle

\clearpage

\clearpage

\section{Introduction}

Certain neurological medical disorders require continuous monitoring to fully understand and diagnose. Examples of medical interest include epilepsy, syncope, multiple sclerosis, migraines, strokes, Parkinson’s and Alzheimer’s disease.  is the recording of electrical activity along the scale, resulting from ionic current flows within the neurons of the brain and is useful for both diagnostic and monitoring such afromentioned conditions.  Amongst other things, monitoring  signals can help physicians understand certain characteristics, triggers, and the severity of the disorder. In particular, it is possible to gauge particular areas of the brain where the condition is originating from and if the patient is a suitable candidate from treatment. \


Patients, however, are unlikely to suffer from neurological disorder while at the clinic as they often appear sporadically during day to day life and with little to no warning. In such circumstances the patient could remain in hospital indefinitely, however seizures can be minutes to years apart, and sometimes are not realised or detected without proper equipment. Such circumstances lend themselves to an ambulatory system, where the patient can be monitored continuously without discomfort or hospitalisation, thus improving quality of life. 

With the plethora of emerging low power wireless technologies coupled with portable devices such as tablets and phones, it is a natural technological step to bring care and monitoring out of the hospital and into the home. The consumer fitness sector is being targeted quite strongly, and many devices already exist that utilise lower power technologies to act as gateways for real-time data logging. For example, there already exists a competitive market between heartbeat monitors, cadence monitors and pedometers. These ‘activity trackers’ use low power electronics and radios to log user’s activities and update the user in real time with activity information through the user’s phone or smart watch. Popular products on the market at the time of writing include the Fitbit, Fuelband and Jawbone, which all use the Bluetooth Low Energy technology to connect to smartphones.

Currently Bluetooth Low Energy (BLE) is the only radio technology that is currently being built into all smart phones and tablet devices while offering power consumption low enough to enable a long lifetimes from a lightweight power source (classic Bluetooth’s power consumption is typically 1 to 2 orders of magnitude higher than BLE). Through leveraging widely popular and familiar smart phone devices with this new technology, in the context of an ambulatory EEG, it is possible to empower the patient to inexpensively take health care into their own home and out of the hospital. 
Further, while this project is targeted at EEG signals, there is no reason why this research and technology cannot be applied to other signals and systems, for example glucose monitoring, electrocardiography, spirometers, etc.
\end{document}